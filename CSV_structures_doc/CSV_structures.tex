\documentclass[a4paper, 16pt]{article}

\usepackage[czech]{babel}
\usepackage[utf8]{inputenc}
\usepackage[left=2cm, top=3cm, text={17cm, 24cm}]{geometry}
\usepackage{pdflscape}
\usepackage{graphicx}
\usepackage{picture}
\usepackage[dvipsnames]{xcolor}
\usepackage{amsthm, amssymb, amsmath}
\usepackage{float}


\begin{document}
    			\begin{center}
				{\LARGE
					Ukládání a příprava dat -- Projekt \\
					{\Large
					    Štruktúry vytvorených CSV súborov \\[5mm]
					}
				}
				{\large
					Matej Berezný, Ondrej Valo, Švenk Adam \\	{\small xberez03, xvaloo00, xsvenk00}
					
				}
			\end{center}
			\vspace{0.7cm}

\section{1. dotaz skupiny A}
V rámci prvej úlohy sa načítavali kolekcie 'hospitalizovany' a 'statistika\_celkovo' v ktorý sa dáta následne združovali podla dátumov po 30 dňoch a využili ako unikátne kľúče. Zanechali sa atribúty v hospitalizovaných: 'pacient\_prvni\_zaznam'.
A v štatistika celkovo: 'prirustkovy\_pocet\_nakazenych', prirustkovy\_pocet\_vylecenych', 'prirustkovy\_pocet\_provedenych\_testu'\\
Následne boli premenované:\\\\
'datum' -$>$ 'date'\\
'prirustkovy\_pocet\_nakazenych' -$>$ 'Positive'\\
'prirustkovy\_pocet\_vylecenych' -$>$ 'Recovered'\\
'prirustkovy\_pocet\_provedenych\_testu' -$>$ 'Tests'\\
'pacient\_prvni\_zaznam' -$>$ 'Hospitalized'\\\\
Nakoniec boli tabuľky spojené pomocou dátumov.
\begin{table}[H]\centering
\begin{tabular}{|l|l|l|}
\hline
\textbf{Date} & Category  & Count  \\ \hline
\end{tabular}
\caption{Štruktúra CSV súboru k 1. dotazu A}
\end{table}


\section{2. dotaz skupiny A}
V druhej úlohe sa využila kolekcia z databáze 'nakazeny\_kraj' z ktorej boli atribúty '\_id', 'vek' a 'kraj\_nuts\_kod' uložené do požadovaného CSV súboru, nepotrebné atribútu sa neukladali, atribút 'vek' sa premenoval na 'Age' a atribút 'kraj\_nuts\_kod' bol premenovaný na 'Region'
\begin{table}[H]\centering
\begin{tabular}{|l|l|l|l|}
\hline
\textbf{\_id} & Age  & Region & Count \\ \hline
\end{tabular}
\caption{Štruktúra CSV súboru k 2. dotazu A}
\end{table}


\section{Dotaz skupiny B}
V dotaze skupiny B boli využité kolekcie: 'obyvatelia' a 'nakazeny\_kraj'.\\\\
Kde v kolekcií obyvatelia sa dátum z dátového typu string pretypoval na date a zachovali sa iba atribúty a niektoré ich hodnoty:\\
Atribút 'vuzemi\_cis' kde hodnoty sa rovnajú '100',\\
Atribút 'vek\_cis' kde hodnoty sa rovnajú '0.0',\\
Atribút 'pohlavi\_cis' kde hodnoty sa rovnajú '0.0',\\
Atribút 'casref\_do' kde hodnota je najaktuálnejší rok',\\
'hodnota' -$>$ 'total\_people'\\\\
A v kolekcií nakazený kraj boli dáta zoskupené podľa dátumu po troch mesiacoch a podla kódu kraja. A brali sa záznami kde kód kraja nebol 0. Premenované atribúty boli:\\
'kraj\_nuts\_kod' -$>$ 'Region'\\
'datum' -$>$ 'Date'\\
Nakoniec bol pridaný stĺpec 'metric' ktorý sa rovná nakazaným za mesiac/celkový počet ľudí.\\\\
Nakoniec obe kolekcie boli pripnuté za seba a uložené do jedného CSV súbora.

\begin{table}[H]\centering
\begin{tabular}{|l|l|l|l|l|l|}
\hline
Region & \textbf{index} & Date & infected\_per\_month & total\_people & metric \\ \hline
\end{tabular}
\caption{Štruktúra CSV súborov k dotazu B}
\end{table}


\section{Vlastný dotaz 1}

V prvom vlastnom dotaze boli využité kolekcie:  'hospitalizovani\_ockovanie', 'zemreli\_ockovanie', 'jip\_ockovanie' z ktorých boli extrahované údaje o očkovaných/neočkovaných pacientoch, ďalej 'ockovanie\_kraj', z ktorej sa získali údaje o očkovaniach v jednotlivých krajoch a tabuľky 'obyvatelia', z ktorej sa získal celkový počet obyvateľov pre jednotlivé kraje za posledný rok. Vo všetkých kolekciách sa dáta zoskupujú podľa dátumov po jednom mesiaci a zároveň sa dátum využíva ako unikátny kľúč. Následne sa stĺpec 'dátum' premenuje na 'date'.\\
V kolekcii obyvatelia sa berú dáta kde kód územia je hodnota 100 (iba kraje), a hodnota stĺpcov vek\_cis a pohlavi\_cis je 0.0 (nezáleží na veku či pohlaví) a kde casref\_do je posledný rok. V kolekcií ockovany\_kraj sa premenuje stĺpec druhych\_davek na Vaccinated \% a prepočíta sa jej hodnota na percentá k pomeru počtu obyvateľov. Nakoniec sa kolekcie spájajú podľa dátumu do dvoch tabuliek, kde jedna je pre nevakcinovaných a druhá pre vakcinovaných. Kde obe berú dáta z viacerých tabuliek. Kde vo finále vo vakcinovanej tabuľke date nám predstavuje dátum Category predstavuje kategóriu ako úmrtie alebo hospitalizácia a percentá počet percent ku počtu obyvateľov. nakoniec je ešte jedna kolekcia predstavujúca preočkovanosť obyvateľstva.

\begin{table}[H]\centering
\begin{tabular}{|l|l|l|}
\hline
\textbf{Date} & Category & Percent \\ \hline
\end{tabular}
\caption{Štruktúra CSV\_0 a CSV\_1 súborov k vlastnému dotazu 1 - Očkovaní/Neočkovaní}
\end{table}

\begin{table}[H]\centering
\begin{tabular}{|l|l|l|}
\hline
\textbf{Date} & Vaccinated \% \\ \hline
\end{tabular}
\caption{Štruktúra CSV\_2 súboru k vlastnému dotazu 1}
\end{table}


\section{Vlastný dotaz 2}
V druhom vlastnom dotaze sa využili kolekcie 'hospitalizovany' a 'statistika\_celkovo'\\
V kolekcii hospitalizovaný boli dáta zoskupené podľa dátumu po jednom mesiaci a takýto dátum bol využitý ako unikátny kľúč.\\
V kolekcii štatistika celkovo bol dátum pretypovaný z string na date a dáta boli zoskupené podla dátumu po jednom mesiaci. A kolekcie boli prepojené pomocov dátumov\\\\
Následne boli niektoré stĺpce a premenné premenované:\\
'datum' -$>$ 'date'\\
'prirustkovy\_pocet\_nakazenych' -$>$ 'New cases'\\
'jip' -$>$ 'Intensive care unit'\\
'kyslik' -$>$ 'Oxygen'\\
'upv' -$>$ 'Artificial lung ventilation'\\
'ecmo' -$>$ 'ECMO'\\
'tezky\_upv\_ecmo' -$>$ 'ALV + ECMO'\\

\begin{table}[H]\centering
\begin{tabular}{|l|l|l|}
\hline
\textbf{Date} & Category & Count \\ \hline
\end{tabular}
\caption{Štruktúra CSV súboru k dotazu B}
\end{table}

\section{Dotaz skupiny C}
V dotaze skupiny C boly použité kolekcie 'obyvatelia', 'obce', 'ockovanie\_zariadenia' a 'ockovaci\_mista. Pomocou kolekcie 'obyvatelia' bolo možné získať dáta týkajúce sa počtu obyvateľov jednotlivých okresov a taktiež aj ich prislúchajúcemu vekovému rozloženiu. Následne boli z kolekcie 'obce' získané údaje o počte infikovaných a tie boli vyfiltrované za časové obdobie poslednéhých 4 štvrťrokov (čiže obdobie posledného roka). Kolekcia 'ockovaci\_mista' slúžila ako zdroj dát týkajúcich sa počtu vykonaných očkovaní spoločne s kolekciou 'ockovanie\_zariadenia', ktorá slúžila na mapovanie vykonaného očkovania v očkovacom zariadení na konkrétny okresný celok. Samotná štruktúra výsledného CSV súboru je nasledovná, pričom zvýraznené atribúty predstavujú atribúty, na ktorých bola vykonaná normalizácia a diskretizácia:

\begin{center}
\begin{table}[H]
\label{tab_c1}
\begin{tabular}{|l|l|l|l|l|l|l|}
\hline
Region & Age [0-15] & Age [15-55] & Age [55+] & Infected & \textbf{Vaccination\_percentage} & \textbf{Vaccination}\\
\hline
\end{tabular}
\caption{Štruktúra CSV súboru k dotazu C}
\end{table}
\end{center}

	
\end{document}
